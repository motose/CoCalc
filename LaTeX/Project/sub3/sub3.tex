%% -*- mode: latex; coding: utf-8; indent-tabs-mode: nil -*-
%%
%% Copyright(C) Yutaka Motose All rights reserved.
%% Lastupdate: 2023-03-20 23:13:57 JST
%%
%% Author: Yutaka Motose <motose@etech21.net>
%%

% ----------------------------------------------------------------------
\documentclass[../master.tex]{subfiles}
% 文書クラスが特殊であることに注意.
% オプション引数に親ファイルへのパスを指定し,クラスをsubfilesとする.

\graphicspath{{images/}}
% sub1.texからみた画像フォルダへのパス

\setcounter{section}{0}
% セクション番号を1から付番(0からではないです)

\begin{document}
\section{サブ3}
    これはsub3.texです.

    コンパイル時に親ファイルのプリアンブルが引き継がれるので親ファイルで設定したマクロも使えます:
    もちろん親ファイルで読み込んだパッケージも使えます.
    \begin{comment}
        コメント環境
    \end{comment}

    画像フォルダにパスを通しているので,
    \begin{figure}[h]
        \centering
        %\includegraphics{sub1/sub1_1.png}
    \end{figure}
    のようにして画像を挿入することができます.

    一応数式を書いてみます.
    %\begin{align}
        y=f(x)
    %\end{align}

    \section{2023年3月21日}
\subsection{初めてのSage\TeX{}}

Sage\TeX{}を使うと、あなたの\LaTeX{}ドキュメントにSageの計算結果を埋め込む事ができます。
$\sage{number_of_partitions(8)}$ integer partitions of $8$.
自分で計算したり、どこかへカットアンドペーストする必要もありません。

以下がSage codeの一例です。

\begin{sageblock}
    f(x) = exp(x) * sin(2*x)
\end{sageblock}

この関数 $f$ の微分は

\[
    \frac{\mathrm{d}^{2}}{\mathrm{d}x^{2}} \sage{f(x)} =
    \sage{diff(f, x, 2)(x)}.
\]

区間 $-1$ から $1$ で $f$ をプロットすると
\sageplot[width=2.5in]{plot(f, -1, 1)}

\begin{sagesilent}
    f(x,y)=x*sin(y); grad_f=f.gradient()
\end{sagesilent}
Let $f(x,y)=\sage{f(x,y)}$. Then $\nabla f=\sage{grad_f(x,y)}$.
\sageplot[width=1.5in]{plot_vector_field(grad_f, (x,-3,3), (y,0,3)),frame=True, aspect_ratio=1}

\begin{sagesilent}
    f(x,y)=x*sin(y)+y*cos(x)
\end{sagesilent}
\sageplot[width=2.5in]{plot3d(f,(x,-3,3),(y,-3,3))}


%\ifSubfilesClassLoaded{
%\bibliographystyle{unsrt}
%\bibliography{ref2.bib}
%}{}

\end{document}
%% ---------------------------------------------------------------------
%% coding: utf-8
%% version-control: t
%% kept-new-versions: 3
%% kept-old-versions: 0
%% End:
%% sub3.tex ends here
